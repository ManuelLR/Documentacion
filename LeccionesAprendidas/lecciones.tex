\documentclass[a4paper,10pt]{scrartcl}
\usepackage[utf8]{inputenc}
\usepackage[spanish]{babel} 
\usepackage[hidelinks]{hyperref}
\usepackage{color}
\usepackage{graphicx}
\usepackage{float}
\graphicspath{ {images/} }


\title{Lecciones Aprendidas}
\subtitle{Grupo 1.2.2 - Redmine}
\author{
		Manuel Francisco López Ruiz\\
		Julio Márquez Castro\\
		Álvaro Martín Gordillo\\
		  }

\begin{document}

\clearpage\maketitle
\thispagestyle{empty}
\newpage




\section{Lecciones aprendidas}

Hemos usado la herramienta Redmine, conocida por algunos integrantes del grupo y desconocida para otros, lo que ha facilitado su aprendizaje. También comentar que es una herramienta muy parecida a Projetsii, con la que el grupo ya esta bastante familiarizado. Hemos usado una herramienta inusual para la creación de los documentos, LaTex, presentada por uno de los integrantes del grupo. Hemos usado Git para mantener un buen control de versiones de nuestros documentos, esto nos ha permitido controlar mejor las versiones de los documentos, pudiendo trabajar cada uno en su propia rama tal y como hemos aprendido en clase.\\\\

Nos ha parecido muy útil la herramienta toggl.com para controlar las horas de trabajo, nos ha permitido aprender de nosotros mismos, mejorar nuestra eficiencia y un mejor distribución del trabajo en grupo. \\\\

Hemos tenido que aprender a manejar herramientas de grabación y edición de vídeo. Esto nos ha retrasado un poco más por la falta de conocimientos en esta materia por parte del grupo. Tras algo de investigación hemos usado Sony Vegas para la creación del vídeo. \\\\

Una pequeña dificultad un poco inesperada es el planteamiento del ejercicio para resolver en clase. Es una tarea tediosa, presentar un ejercicio de acuerdo a la herramienta a presentar, ajustándose al tiempo establecido e incluyendo los aspectos mas importantes. \\\\

En general hemos aprendido de una nueva metodología de trabajo, adaptada a enseñar al resto de nuestros compañeros una herramienta software. Nos ha planteado cuestiones y retos nuevos, que quizás no nos planteábamos al empezar el ejercicio. 


\bibliography{sample}

\end{document}
