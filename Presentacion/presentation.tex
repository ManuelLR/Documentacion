\documentclass[
xcolor={svgnames},
hyperref={colorlinks,citecolor=DeepPink4,linkcolor=Black,urlcolor=DarkBlue}
]{beamer}
\usepackage[utf8]{inputenc}
\usepackage[spanish]{babel} 
\usepackage{media9}

\usepackage{multimedia}
\usepackage{hyperref}
\usepackage{graphicx}

\usepackage{setspace}
\usepackage{xcolor}
%\usepackage{listings}
%\usepackage{minted}



\usetheme{CambridgeUS}
\usecolortheme{seahorse} % https://www.hartwork.org/beamer-theme-matrix/
% \useoutertheme{shadow}
\useinnertheme{rectangles}

\AtBeginSection[]
{
	\begin{frame}<beamer>
		\frametitle{Índice}
		\tableofcontents[
			currentsection
		]
	\end{frame}
}


\title{Redmine}
\subtitle{Grupo 1.2.2}
\author[Grupo 1.2.2]{
	Manuel Francisco López Ruiz\\
	Julio Márquez Castro\\
	Álvaro Martín Gordillo\\
}
\institute[PGPI]{
	Planificación y Gestión de Proyectos Informáticos [PGPI] \\
	Grado en Ingeniería Informática - Ingeniería del Software
	\and
	Universidad de Sevilla (Spain)
}

\begin{document}

\begin{frame}[plain]
	% Run with pdflatex --shell-escape 
	\titlepage
\end{frame}



\begin{frame}
	\frametitle{Índice}
	\tableofcontents
\end{frame}


\section{Introducción} 
\subsection{Descripción del trabajo realizado}

\begin{frame}
	\frametitle{Descripción del trabajo realizado}
	%GitHub
	%LaTex
	%Ganttproject

\end{frame}

\subsection{Lecciones aprendidas}
\begin{frame}
	\frametitle{Lecciones aprendidas}
\end{frame}

\section{Presentación de Redmine}

\subsection{Vídeo explicativo}

\begin{frame}
	\begin{center}
		\movie[height = 0.6\textwidth, width = 0.8\textwidth, poster, showcontrols] {}{Video_PGPI.mp4}
	\end{center}

\end{frame}

\subsection{¿Otras aclaraciones?}

\begin{frame}
	¿ tendremos que aclarar algo más que no diera tiempo en el vídeo?
\end{frame}

\begin{frame}
	\begin{center}

		{\Large 		\textbf{¿Alguna pregunta sobre Redmine antes de realizar el ejercicio práctico?}}

	\end{center}

\end{frame}

\section{Ejercicio práctico}

\subsection{Enunciado}
\begin{frame}

\end{frame}

\subsection{Resolución}

\begin{frame}

\end{frame}

\section{Preguntas}

\begin{frame}
	\begin{center}
		{\Large 		\textbf{¿Alguna pregunta?}}
	\end{center}
\end{frame}

%\begin{frame}[allowframebreaks]
%	\frametitle{Referencias}
%
%	\bibliographystyle{plain}
%	\bibliography{presentation} 
%
%
%\end{frame}

\begin{frame}[plain]
	\titlepage
\end{frame}

\end{document}
