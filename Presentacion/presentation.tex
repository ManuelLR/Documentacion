\documentclass[
xcolor={svgnames},
hyperref={colorlinks,citecolor=DeepPink4,linkcolor=Black,urlcolor=DarkBlue}
]{beamer}
\usepackage[utf8]{inputenc}
\usepackage[spanish]{babel} 
\usepackage{media9}

\usepackage{multimedia}
\usepackage{hyperref}
\usepackage{graphicx}

\usepackage{setspace}
\usepackage{xcolor}
%\usepackage{listings}
%\usepackage{minted}



\usetheme{CambridgeUS}
\usecolortheme{seahorse} % https://www.hartwork.org/beamer-theme-matrix/
% \useoutertheme{shadow}
\useinnertheme{rectangles}

\AtBeginSection[]
{
	\begin{frame}<beamer>
		\frametitle{Índice}
		\tableofcontents[
			currentsection
		]
	\end{frame}
}


\title{Redmine}
\subtitle{Grupo 1.2.2}
\author[Grupo 1.2.2]{
	Manuel Francisco López Ruiz\\
	Julio Márquez Castro\\
	Álvaro Martín Gordillo\\
}
\institute[PGPI]{
	Planificación y Gestión de Proyectos Informáticos [PGPI] \\
	Grado en Ingeniería Informática - Ingeniería del Software
	\and
	Universidad de Sevilla (Spain)
}

\begin{document}

\begin{frame}[plain]
	% Run with pdflatex --shell-escape 
	\titlepage
\end{frame}



\begin{frame}
	\frametitle{Índice}
	\tableofcontents
\end{frame}


\section{Introducción} 
\begin{frame}
		\begin{itemize}
			\item \href{http://www.redmine.org}{Redmine}
			% Conocido por algunos integrantes. Familiarizado por la similitud con projetsii.
			\item \LaTeX
			\item \href{https://github.com/Redmine-PGPI/Documentacion}{GitHub}
			% Poder usar Git. Cierta despreocupacion visual.
			\item \href{https://toggl.com}{Toggl}
			% Mejorar Eficiencia y reparto de carga de trabajo.
			\item \href{http://www.ganttproject.biz}{Ganttproject}
			% Para realizar la planificación en lugar de Microsoft Project
			\item Edición de vídeo
			% Desconocimiento total por parte del equipo de trabajo.
			\item Metodología de trabajo
			% Nueva metodología de trabajo orientada a enseñar a los compañeros.
		\end{itemize}
\end{frame}

\section{Presentación de Redmine}

\subsection{Vídeo explicativo}

\begin{frame}
	\begin{center}
%		\movie[height = 0.6\textwidth, width = 0.8\textwidth, poster, showcontrols] {
			\href{https://youtu.be/aCrP9A_wCWE}{YouTube}
%		}{Video_PGPI.mp4}
	\end{center}

\end{frame}

\begin{frame}
	\begin{center}

		{\Large 		\textbf{¿Alguna pregunta sobre Redmine antes de realizar el ejercicio práctico?}}

	\end{center}

\end{frame}

\section{Ejercicio práctico}

\subsection{Enunciado}
\begin{frame}

\end{frame}

\subsection{Resolución}

\begin{frame}

\end{frame}

\section{Preguntas}

\begin{frame}
	\begin{center}
		{\Large 		\textbf{¿Alguna pregunta?}}
	\end{center}
\end{frame}

%\begin{frame}[allowframebreaks]
%	\frametitle{Referencias}
%
%	\bibliographystyle{plain}
%	\bibliography{presentation} 
%
%
%\end{frame}

\begin{frame}[plain]
	\titlepage
\end{frame}

\end{document}
