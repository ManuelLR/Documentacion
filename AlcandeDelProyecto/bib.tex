\documentclass[a4paper,10pt]{scrartcl}
\usepackage[utf8]{inputenc}
\usepackage[spanish]{babel} 
\usepackage[hidelinks]{hyperref}



\title{Alcance del proyecto}
\subtitle{Grupo 1.2.2 - Redmine}
\author{
		Manuel Francisco López Ruiz\\
		Julio Márquez Castro\\
		Álvaro Martín Gordillo\\
		  }

\begin{document}

\clearpage\maketitle
\thispagestyle{empty}
\newpage

\tableofcontents

\newpage



\section{Objetivos del proyecto}

Obtener un alto nivel de conocimientos sobre la herramienta Redmine con el fin de ser capaces de transmitir los conocimientos aprendidos.

\section{Descripción del producto o servicio}

El servicio es un curso introductorio a Redmine donde se debe explicar que es, como se usa, errores comunes y ventajas e inconvenientes frente a otros servicios. Tras la explicación se realizará una práctica con el objetivo de que los asistentes puedan practicar los conceptos aprendidos y preguntar dudas.

\section{Requisitos del proyecto/producto}

\begin{itemize}
	\item Software Redmine
	\item Acceso a internet
	\item Servidor para Redmine
	\item Pc cliente
\end{itemize}

\section{Límites/restricciones del proyecto}

\begin{itemize}
	\item Limite de tiempo del vídeo tutorial
\end{itemize}

\section{Descripción del producto o servicio}

\section{Requisitos del proyecto/producto}

\section{Límites/restricciones del proyecto}

\section{Entregables}

\section{Criterios de aceptación}

\section{Exclusiones}

\section{Asunciones}

\section{Riesgos iniciales}

\section{Fechas importantes}


\section{EDT}

\section{Diccionario EDT}



\section{Estimación de costes}


\bibliography{sample}

\end{document}
