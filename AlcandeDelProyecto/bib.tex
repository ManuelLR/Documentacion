\documentclass[a4paper,10pt]{article}
\usepackage[utf8]{inputenc}

\title{Alcance del proyecto}
\author{autoresssss}

\begin{document}

\maketitle



\section*{Objetivos del proyecto}

\paragraph{Debemos obtener un alto nivel de conocimientos de la herramienta Redmine, amplia soltura en su manejo y corrección de errores, para poder ser capaces de enseñar los conocimientos aprendidos.}

\section*{Descripción del producto o servicio}

\paragraph{El servicio es un tutorial de Redmine, donde se explique que es, como es usa, errores comunes, ventajas e inconvenientes contra otros servicios. Interactuando con los alumnos del tutorial, dando soporte y acceso a la herramienta con el objetivo de que puedan practicar los conceptos aprendidos.}

\section*{Requisitos del proyecto/producto}

\begin{enumerate}
	\item Software Redmine
	\item Acceso a internet
	\item Servidor para Redmine
	\item Pc cliente
\end{enumerate}

\section*{Límites/restricciones del proyecto}

\begin{enumerate}
	\item Limite de tiempo del vídeo tutorial
\end{enumerate}

%\subtitle{Descripción del producto o servicio}

%\subtitle{Requisitos del proyecto/producto}

%\subtitle{Límites/restricciones del proyecto}

%\subtitle{Entregables}

%\subtitle{Criterios de aceptación}

%\subtitle{Exclusiones}

%\subtitle{Asunciones}

%\subtitle{Riesgos iniciales}

%\subtitle{Fechas importantes}

%\subtitle{Estimación de costes}


\bibliography{sample}

\end{document}
