\documentclass[a4paper,10pt]{scrartcl}
\usepackage[utf8]{inputenc}
\usepackage[spanish]{babel}
\usepackage{booktabs}
\usepackage[table,xcdraw]{xcolor}
\usepackage{graphicx}
\usepackage{enumitem}
\usepackage[hidelinks]{hyperref}
\usepackage{blindtext}
\usepackage{array}
\newcolumntype{M}[1]{>{\centering\arraybackslash}m{#1}}

\title{Acta de Constitución}
\subtitle{Grupo 1.2.2 - Redmine}
\author{Manuel Francisco López Ruiz \\
	Julio Márquez Castro \\
	 Álvaro Martín Gordillo}
	


\begin{document}

\clearpage\maketitle
\thispagestyle{empty}



\begin{center}
	\begin{table}
	\centering
	\begin{tabular}{| c | M{2cm} | M{2cm} | M{2cm} | c | M{3cm} |}
		\hline
		\multicolumn{6}{|c|}{\textbf{Control de versiones}} \\ \hline
		\textbf{Ver.} & \textbf{Hecha por} & \textbf{Revisada por} & \textbf{Aprobada por} & \textbf{Fecha} & \textbf{Motivo} \\ \hline
		0.1 & Álvaro Martín Gordillo & Manuel Francisco López Ruiz & -- & 28/10/2016 & Creación del acta \\ \hline
	\end{tabular}
	\end{table}
\end{center}

\newpage

\tableofcontents


\newpage
\section{Descripción del proyecto}


		El proyecto consiste en la creación de un videotutorial y de un ejercicio práctico para exponer el uso de la herramienta de RedMine al resto del grupo de PGPI.
		Esto será realizado mediante la herramienta de grabado de alguna de las máquinas virtuales que estamos usando.
		La puesta en práctica del mismo se realizará a partir del 12 de diciembre en el horario que se asigne en la clase.
		
\section{Definición del producto del proyecto}

		El producto será el resultado de enseñar el videotutorial y la realización de las actividades con el propósito de enseñar al resto de la clase cómo usar la herramienta de RedMine y por qué es útil en un entorno de desarrollo.
		
\section{Definición de requisitos del proyecto}
		
		Requisitos
		
		
\section{Objetivos del proyecto}
\subsection{Alcance}


\subsection{Tiempo}		
		
		
\subsection{Costo}



\section{Finalidad del proyecto}

	Aprobar la asignatura de PGPI con la calificación nota posible a la vez que conseguir que nuestros compañeros aprendan sobre RedMine
	
\section{Justificación del proyecto}
\subsection{Justificación cualitativa}

\subsection{Justificacion cuantitativa}
\textcolor{red} {Qué hacemos con esto}


\begin{description}[align=right, labelwidth=4cm]
	\item [Flujo de ingresos] description
	\item [Flujo de egresos] description
	\item [VAN] description
	\item [TIR] description
	\item [RBC] description
\end{description}


\section{Designación del Project Manager del proyecto}

\begin{description}[align=right, labelwidth=4cm]
	\item [Nombre] Manuel Francisco López Ruiz
	\item [Reporta a] Pablo Trinidad
	\item [Supervisa a] Julio Márquez Castro
	\item [Supervisa a] Álvaro Martín Gordilo
\end{description}


\section{Cronograma e Hitos del proyecto}

\begin{center}
	\begin{tabular}{| l | l |}
		\hline	
		\textbf{Hito o evento significativo} & \textbf{Fecha programada} \\ \hline
 		Entrega del proyecto & 7 de Diciembre 2016 \\ \hline
	\end{tabular}
\end{center}

\section{Organizaciones o grupos organizacionales que intervienen en el proyecto}
\begin{center}
	\begin{tabular}{| l | l |}
		\hline	
		\textbf{Organización o grupo organizacional} & \textbf{Rol que desempeña} \\ \hline
		Departamento de Lenguajes y Sistemas Informáticos & Evaluador \\ \hline
		Grupo 1.2.2 - Redmine & Desarrolladores \\ \hline
	\end{tabular}
\end{center}

\section{Principales amenazas del proyexto}
\begin{itemize}
	\item Miedo escénico
	\item Github dejando de funcionar
	\item El servidor de RedMine caído
\end{itemize}

\section{Principales oportunidades del proyexto}
\begin{itemize}
	\item Miedo escénico
	\item Github dejando de funcionar
	\item El servidor de RedMine caído
\end{itemize}

\section{Sponsor que autoriza el proyecto}
\begin{itemize}
	\item Pablo Trinidad - LSI - Profesor  - 7/10/2016
\end{itemize}

%\begin{table}[]
%	\centering
%	\label{my-label}
%	\resizebox{\textwidth}{!}{%
%		\begin{tabular}{@{}|c|@{}}
%			\toprule
%			\rowcolor[HTML]{9B9B9B} 
%			\textbf{Descripción del proyecto: \textsc{¿Qué, quién, cómo, cuándo y dónde?}} \\ \midrule
%			\rowcolor[HTML]{FFFFFF} 
%			XX \\ \midrule
%			\rowcolor[HTML]{9B9B9B} 
%			\textbf{Definición del producto del proyecto: \textsc{Descripción del producto, servicio o capacidad a generar}} \\ \midrule
%			\\ \midrule
%			\rowcolor[HTML]{9B9B9B} 
%			\textbf{Definición de requisitos del proyecto: \textsc{Descripción de requerimientos funcionales, no funcionales, de calidad, etc., del proyecto/producto}} \\ \midrule
%			\multicolumn{1}{|l|}{} \\ \bottomrule
%		\end{tabular}%
%	}
%\end{table}cm



%\begin{table}[]
%	\centering
%	\label{my-label}
%	\resizebox{\textwidth}{!}{%
%		\begin{tabular}{@{}cll@{}}
%			\toprule
%			\rowcolor[HTML]{9B9B9B} 
%			\multicolumn{3}{c}{\cellcolor[HTML]{9B9B9B}\textbf{OBJETIVOS DEL PROYECTO: \textsc{METAS HACIA LAS CUALES SE DEBE DIRIGIR EL TRABAJO DEL PROYECTO EN TÉRMINOS DELA TRIPLE RESTRICCIÓN.}}} \\ \midrule
%			\rowcolor[HTML]{C0C0C0} 
%			\multicolumn{1}{|c|}{\cellcolor[HTML]{C0C0C0}\textbf{Concepto}} & \multicolumn{1}{c|}{\cellcolor[HTML]{C0C0C0}\textbf{Objetivos}} & \multicolumn{1}{c|}{\cellcolor[HTML]{C0C0C0}\textbf{Criterio de éxito}} \\ \midrule
%			\multicolumn{1}{|c|}{\cellcolor[HTML]{EFEFEF}\textbf{1. Alcance}} & \multicolumn{1}{l|}{} & \multicolumn{1}{l|}{} \\ \midrule
%			\multicolumn{1}{|c|}{\cellcolor[HTML]{EFEFEF}\textbf{2. Costo}} & \multicolumn{1}{l|}{} & \multicolumn{1}{l|}{} \\ \midrule
%			\multicolumn{1}{|c|}{\cellcolor[HTML]{EFEFEF}\textbf{3. Tiempo}} & \multicolumn{1}{l|}{} & \multicolumn{1}{l|}{} \\ \midrule
%			\rowcolor[HTML]{9B9B9B} 
%			\multicolumn{3}{|c|}{\cellcolor[HTML]{9B9B9B}\textbf{FINALIDAD DEL PROYECTO: \textsc{FIN ÚLTIMO, PROPÓSITO GENERAL, U OBJETIVO DE NIVEL SUPERIOR POR EL CUAL SE EJECUTA EL PROYECTO. ENLACE CON PROGRAMAS, PORTAFOLIOS, O ESTRATEGIAS DE LA ORGANIZACIÓN.}}} \\ \midrule
%			\multicolumn{1}{|l|}{} & \multicolumn{1}{l|}{} & \multicolumn{1}{l|}{} \\ \midrule
%			\rowcolor[HTML]{9B9B9B} 
%			\multicolumn{3}{|c|}{\cellcolor[HTML]{9B9B9B}\textbf{JUSTIFICACIÓN DEL PROYECTO: \textsc{MOTIVOS, RAZONES, O ARGUMENTOS QUE JUSTIFICAN LA EJECUCIÓN DEL PROYECTO.}}} \\ \midrule
%			\multicolumn{1}{|c|}{\textbf{JUSTIFICACIÓN CUALITATIVA}} & \multicolumn{2}{c|}{\textbf{JUSTIFICACIÓN CUANTITATIVA}} \\ \midrule
%			\multicolumn{1}{|l|}{} & \multicolumn{1}{l|}{Flujo de ingresos} & \multicolumn{1}{l|}{} \\ \midrule
%			\multicolumn{1}{|l|}{} & \multicolumn{1}{l|}{Flujo de egresos} & \multicolumn{1}{l|}{} \\ \midrule
%			\multicolumn{1}{|l|}{} & \multicolumn{1}{l|}{VAN} & \multicolumn{1}{l|}{} \\ \midrule
%			\multicolumn{1}{|l|}{} & \multicolumn{1}{l|}{TIR} & \multicolumn{1}{l|}{} \\ \midrule
%			\multicolumn{1}{|l|}{} & \multicolumn{1}{l|}{RBC} & \multicolumn{1}{l|}{} \\ \bottomrule
%		\end{tabular}%
%	}
%\end{table}

%\begin{table}[]
%	\centering
%	\caption{My caption}
%	\label{my-label}
%	\begin{tabular}{@{}
%			>{\columncolor[HTML]{C0C0C0}}l ll@{}}
%		\multicolumn{3}{c}{\cellcolor[HTML]{9B9B9B}\textbf{DESIGNACIÓN DEL PRODUCT MANAGER DEL PROYECTO}} \\
%		\textbf{NOMBRE} &  & \multicolumn{1}{c}{\cellcolor[HTML]{C0C0C0}\textbf{NIVELES DE AUTORIDAD}} \\
%		\textbf{REPORTA A} &  &  \\
%		\textbf{SUPERVISA A} &  & \multirow{-2}{*}{}
%	\end{tabular}
%\end{table}

\end{document}
