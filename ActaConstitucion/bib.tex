\documentclass[a4paper,10pt]{scrartcl}
\usepackage[utf8]{inputenc}
\usepackage[spanish]{babel}
\usepackage{booktabs}
\usepackage[table,xcdraw]{xcolor}
\usepackage{graphicx}
\usepackage{enumitem}
\usepackage[hidelinks]{hyperref}
\usepackage{blindtext}
\usepackage{array}
\newcolumntype{M}[1]{>{\centering\arraybackslash}m{#1}}

\title{Acta de Constitución}
\subtitle{Grupo 1.2.2 - Redmine}
\author{Manuel Francisco López Ruiz \\
	Julio Márquez Castro \\
	 Álvaro Martín Gordillo}
	


\begin{document}

\clearpage\maketitle
\thispagestyle{empty}



\begin{center}
	\begin{table}
	\centering
	\begin{tabular}{| c | M{2cm} | M{2cm} | M{2cm} | c | M{3cm} |}
		\hline
		\multicolumn{6}{|c|}{\textbf{Control de versiones}} \\ \hline
		\textbf{Ver.} & \textbf{Hecha por} & \textbf{Revisada por} & \textbf{Aprobada por} & \textbf{Fecha} & \textbf{Motivo} \\ \hline
		0.1 & Álvaro Martín Gordillo & Manuel Francisco López Ruiz & -- & 28/10/2016 & Creación del acta \\ \hline
		1.0 & Álvaro Martín Gordillo & -- & -- & 24/11/2016 & Finalización del acta \\ \hline
		1.1 & Manuel Francisco López Ruiz & -- & -- & 4/12/2016 & Renombrado de secciones \\ \hline
	\end{tabular}
	\end{table}
\end{center}

\newpage

\tableofcontents


\newpage
\section{Descripción}


		El proyecto consiste en la creación de un videotutorial y de un ejercicio práctico para exponer el uso de la herramienta de RedMine al resto del grupo de PGPI.
		Esto será realizado mediante la herramienta de grabado de alguna de las máquinas virtuales que estamos usando.
		La puesta en práctica del mismo se realizará a partir del 12 de diciembre en el horario que se asigne en la clase.
		
\section{Definición del producto}

		El producto será el resultado de enseñar el videotutorial y la realización de las actividades con el propósito de enseñar al resto de la clase cómo usar la herramienta de RedMine y por qué es útil en un entorno de desarrollo.
		
\section{Definición de requisitos}
	\begin{itemize}
			\item El proyecto debe incluir el plan de gestión del trabajo
			\item El proyecto deberá ser presentado en un plazo de 35 minutos
			\item El proyecto debe incluir un ejercicio práctico dirigido
	\end{itemize}

\section{Objetivos}
\subsection{Alcance}
		El alcance del proyecto consiste en realizar una presentación de la tecnología redmine que sea suficientemente explicativa como para que el resto de alumnos del grupo de prácticas del día en el que se presente lo puedan entender.

\subsection{Tiempo}		
		Para realizar el proyecto hay un plazo de dos meses aproximadamente en el que se estima que puede haber unas 15 horas de tiempo invertido por miembro		
		
\subsection{Costo}
		En cuanto al costo, tenemos que tener en cuenta el precio de nuestros portátiles y su amortación, además de las horas invertidas, lo cual, dependiendo de a cuantos años se estén amortizando los equipos, podría ser alrededor de 500eur  por cada miembro participante.
		Al usar software libre para la realización de cada una de las partes el coste de amortización del software es 0.
		Pero teniendo en cuenta que es un proyecto con fines educativos en el marco estudiantil, el coste asumido es simplemente el costo de amortización de nuestros dispositivos (alrededor de 80eur para el tiempo empleado en el proyecto por cada portátil).


\section{Finalidad}

	Aprobar la asignatura de PGPI con la mayor calificación posible a la vez que conseguir que nuestros compañeros aprendan sobre RedMine.
	
\section{Justificación}

	La justificación del proyecto consiste en que todos los alumnos de la asignatura seamos capaces de usar, o al menos hayamos sido introducidos a ellos, al final del curso muchas de las herramientas comunes de gestión de proyectos de software.


\section{Designación del Project Manager}

\begin{description}[align=right, labelwidth=4cm]
	\item [Nombre] Manuel Francisco López Ruiz
	\item [Reporta a] Pablo Trinidad
	\item [Supervisa a] Julio Márquez Castro
	\item [Supervisa a] Álvaro Martín Gordilo
\end{description}


\section{Cronograma e Hitos}

\begin{center}
	\begin{tabular}{| l | l |}
		\hline	
		\textbf{Hito o evento significativo} & \textbf{Fecha programada} \\ \hline
		Primera revisión del proyecto & 28 de Octubre 2016 \\ \hline
		Segunda revisión del proyecto & 2 de Diciembre de 2016 \\ \hline
 		Entrega del proyecto & 7 de Diciembre 2016 \\ \hline
	\end{tabular}
\end{center}

\section{Organizaciones que intervienen}
\begin{center}
	\begin{tabular}{| l | l |}
		\hline	
		\textbf{Organización o grupo organizacional} & \textbf{Rol que desempeña} \\ \hline
		Departamento de Lenguajes y Sistemas Informáticos & Organizador \\ \hline
		Jose María García Rodríguez & Evaluador \\ \hline
		Pablo Trinidad & Evaluador \\ \hline
		Grupo 1.2.2 - Redmine & Desarrolladores \\ \hline
	\end{tabular}
\end{center}

\section{Principales amenazas}
\begin{itemize}
	\item Miedo escénico
	\item Github dejando de funcionar
	\item El servidor de RedMine caído
	\item No saber usar ninguna herramienta de edición de vídeo
\end{itemize}

\section{Principales oportunidades}
\begin{itemize}
	\item Posibilidad de hacer un vídeo que sorprenda al resto del alumnado así como a los evaluadores del proyecto
	\item Realizar un ejercicio práctico interesante e interactivo
	\item Ser capaces de responder a todas las preguntas que se planteen
\end{itemize}

\section{Sponsor que autoriza}
\begin{itemize}
	\item Pablo Trinidad - LSI - Profesor  - 7/10/2016
	\item Jose María García Rodríguez - LSI - Profesor - 7/10/2016
\end{itemize}

%\begin{table}[]
%	\centering
%	\label{my-label}
%	\resizebox{\textwidth}{!}{%
%		\begin{tabular}{@{}|c|@{}}
%			\toprule
%			\rowcolor[HTML]{9B9B9B} 
%			\textbf{Descripción del proyecto: \textsc{¿Qué, quién, cómo, cuándo y dónde?}} \\ \midrule
%			\rowcolor[HTML]{FFFFFF} 
%			XX \\ \midrule
%			\rowcolor[HTML]{9B9B9B} 
%			\textbf{Definición del producto del proyecto: \textsc{Descripción del producto, servicio o capacidad a generar}} \\ \midrule
%			\\ \midrule
%			\rowcolor[HTML]{9B9B9B} 
%			\textbf{Definición de requisitos del proyecto: \textsc{Descripción de requerimientos funcionales, no funcionales, de calidad, etc., del proyecto/producto}} \\ \midrule
%			\multicolumn{1}{|l|}{} \\ \bottomrule
%		\end{tabular}%
%	}
%\end{table}cm



%\begin{table}[]
%	\centering
%	\label{my-label}
%	\resizebox{\textwidth}{!}{%
%		\begin{tabular}{@{}cll@{}}
%			\toprule
%			\rowcolor[HTML]{9B9B9B} 
%			\multicolumn{3}{c}{\cellcolor[HTML]{9B9B9B}\textbf{OBJETIVOS DEL PROYECTO: \textsc{METAS HACIA LAS CUALES SE DEBE DIRIGIR EL TRABAJO DEL PROYECTO EN TÉRMINOS DELA TRIPLE RESTRICCIÓN.}}} \\ \midrule
%			\rowcolor[HTML]{C0C0C0} 
%			\multicolumn{1}{|c|}{\cellcolor[HTML]{C0C0C0}\textbf{Concepto}} & \multicolumn{1}{c|}{\cellcolor[HTML]{C0C0C0}\textbf{Objetivos}} & \multicolumn{1}{c|}{\cellcolor[HTML]{C0C0C0}\textbf{Criterio de éxito}} \\ \midrule
%			\multicolumn{1}{|c|}{\cellcolor[HTML]{EFEFEF}\textbf{1. Alcance}} & \multicolumn{1}{l|}{} & \multicolumn{1}{l|}{} \\ \midrule
%			\multicolumn{1}{|c|}{\cellcolor[HTML]{EFEFEF}\textbf{2. Costo}} & \multicolumn{1}{l|}{} & \multicolumn{1}{l|}{} \\ \midrule
%			\multicolumn{1}{|c|}{\cellcolor[HTML]{EFEFEF}\textbf{3. Tiempo}} & \multicolumn{1}{l|}{} & \multicolumn{1}{l|}{} \\ \midrule
%			\rowcolor[HTML]{9B9B9B} 
%			\multicolumn{3}{|c|}{\cellcolor[HTML]{9B9B9B}\textbf{FINALIDAD DEL PROYECTO: \textsc{FIN ÚLTIMO, PROPÓSITO GENERAL, U OBJETIVO DE NIVEL SUPERIOR POR EL CUAL SE EJECUTA EL PROYECTO. ENLACE CON PROGRAMAS, PORTAFOLIOS, O ESTRATEGIAS DE LA ORGANIZACIÓN.}}} \\ \midrule
%			\multicolumn{1}{|l|}{} & \multicolumn{1}{l|}{} & \multicolumn{1}{l|}{} \\ \midrule
%			\rowcolor[HTML]{9B9B9B} 
%			\multicolumn{3}{|c|}{\cellcolor[HTML]{9B9B9B}\textbf{JUSTIFICACIÓN DEL PROYECTO: \textsc{MOTIVOS, RAZONES, O ARGUMENTOS QUE JUSTIFICAN LA EJECUCIÓN DEL PROYECTO.}}} \\ \midrule
%			\multicolumn{1}{|c|}{\textbf{JUSTIFICACIÓN CUALITATIVA}} & \multicolumn{2}{c|}{\textbf{JUSTIFICACIÓN CUANTITATIVA}} \\ \midrule
%			\multicolumn{1}{|l|}{} & \multicolumn{1}{l|}{Flujo de ingresos} & \multicolumn{1}{l|}{} \\ \midrule
%			\multicolumn{1}{|l|}{} & \multicolumn{1}{l|}{Flujo de egresos} & \multicolumn{1}{l|}{} \\ \midrule
%			\multicolumn{1}{|l|}{} & \multicolumn{1}{l|}{VAN} & \multicolumn{1}{l|}{} \\ \midrule
%			\multicolumn{1}{|l|}{} & \multicolumn{1}{l|}{TIR} & \multicolumn{1}{l|}{} \\ \midrule
%			\multicolumn{1}{|l|}{} & \multicolumn{1}{l|}{RBC} & \multicolumn{1}{l|}{} \\ \bottomrule
%		\end{tabular}%
%	}
%\end{table}

%\begin{table}[]
%	\centering
%	\caption{My caption}
%	\label{my-label}
%	\begin{tabular}{@{}
%			>{\columncolor[HTML]{C0C0C0}}l ll@{}}
%		\multicolumn{3}{c}{\cellcolor[HTML]{9B9B9B}\textbf{DESIGNACIÓN DEL PRODUCT MANAGER DEL PROYECTO}} \\
%		\textbf{NOMBRE} &  & \multicolumn{1}{c}{\cellcolor[HTML]{C0C0C0}\textbf{NIVELES DE AUTORIDAD}} \\
%		\textbf{REPORTA A} &  &  \\
%		\textbf{SUPERVISA A} &  & \multirow{-2}{*}{}
%	\end{tabular}
%\end{table}

\end{document}
